\documentclass{article}
\nonstopmode



%%% Packages %%%%%%%%%%%%%%%%%%%%%%%%%%%%%%%%%%%%%%%%%%%%%%%%%%%%%%%%%%%%%%%%%%%

\RequirePackage{textcomp}
\RequirePackage{latexsym}
\RequirePackage{amstext}
\RequirePackage{amsmath}
\RequirePackage{amssymb}
\RequirePackage{amsthm}
\usepackage{fullpage}
\usepackage[hidelinks]{hyperref}



%%% Theorem types %%%%%%%%%%%%%%%%%%%%%%%%%%%%%%%%%%%%%%%%%%%%%%%%%%%%%%%%%%%%%%

\theoremstyle{plain}
\newtheorem{theorem}{Theorem}[section]
\newtheorem{lemma}[theorem]{Lemma}
\newtheorem{corollary}[theorem]{Corollary}
\newtheorem{fact}[theorem]{Fact}



%%% Typography %%%%%%%%%%%%%%%%%%%%%%%%%%%%%%%%%%%%%%%%%%%%%%%%%%%%%%%%%%%%%%%%%

\def\labelitemi{\textbullet}
\def\labelitemii{\textopenbullet}
\def\labelitemiii{\textbullet}
\def\labelitemiv{\textopenbullet}



%%% Standard definitions and commands. Link or copy file into directory. %%%%%%%




%%% Definitions %%%%%%%%%%%%%%%%%%%%%%%%%%%%%%%%%%%%%%%%%%%%%%%%%%%%%%%%%%%%%%%%

\def\ep{\varepsilon}

\def\m{\!-\!}
\def\pl{\!+\!}
\def\eq{\!=\!}

\def\tsum{\textstyle\sum}
\def\tsqrt{\textstyle\sqrt}
\def\Var{\text{Var}}

    %%% Modified Latin letters %%%%%%%%%%%%%%%%%%%%%%%%%%%%%%%%%%%%%%%%%%%%%%%%%

\def\Dcal{\mathcal{D}}
\def\Fcal{\mathcal{F}}

\def\Fhat{\hat{F}}
\def\chat{\hat{c}}
\def\dhat{\hat{d}}
\def\fhat{\hat{f}}
\def\ghat{\hat{g}}
\def\nhat{\hat{n}}
\def\that{\hat{t}}

\def\ctl{\tilde{c}}
\def\gtl{\tilde{g}}
\def\htl{\tilde{h}}



%%% Commands %%%%%%%%%%%%%%%%%%%%%%%%%%%%%%%%%%%%%%%%%%%%%%%%%%%%%%%%%%%%%%%%%%%

    %%% Formatting %%%%%%%%%%%%%%%%%%%%%%%%%%%%%%%%%%%%%%%%%%%%%%%%%%%%%%%%%

\newcommand{\p}[1]{(#1)}
\newcommand{\pa}[1]{\langle#1\rangle}
\newcommand{\pb}[1]{[#1]}
\newcommand{\pc}[1]{\{#1\}}

\newcommand{\bp}[1]{\left(#1\right)}
\newcommand{\bpa}[1]{\left\langle#1\right\rangle}
\newcommand{\bpb}[1]{\left[#1\right]}
\newcommand{\bpc}[1]{\left\{#1\right\}}

\newcommand{\ceil}[1]{\lceil#1\rceil}
\newcommand{\floor}[1]{\lfloor#1\rfloor}

\newcommand{\bceil}[1]{\left\lceil#1\right\rceil}
\newcommand{\bfloor}[1]{\left\lfloor#1\right\rfloor}

    %%% General math blocks %%%%%%%%%%%%%%%%%%%%%%%%%%%%%%%%%%%%%%%%%%%%%%%%

\newcommand{\abs}[1]{|#1|}
\newcommand{\set}[1]{\pc{#1}}
\newcommand{\setc}[2]{\pc{#1 : #2}}
\newcommand{\setcsep}[3]{\pc{#1 #3 #2}}

\newcommand{\expp}[1]{\exp\p{#1}}
\newcommand{\logp}[1]{\log\p{#1}}

    %%% Probability %%%%%%%%%%%%%%%%%%%%%%%%%%%%%%%%%%%%%%%%%%%%%%%%%%%%%%%%

\renewcommand{\P}[1]{P\p{#1}}
\newcommand{\E}[1]{E\p{#1}}

    %%% CS theory %%%%%%%%%%%%%%%%%%%%%%%%%%%%%%%%%%%%%%%%%%%%%%%%%%%%%%%%%%

\newcommand{\OO}[1]{O\p{#1}}
\newcommand{\OM}[1]{\Omega\p{#1}}
\newcommand{\OT}[1]{\Theta\p{#1}}

\newcommand{\poly}[1]{\text{poly}\p{#1}}



% Copyright 2016 David Felber.




%%% Document-specific definitions and commands %%%%%%%%%%%%%%%%%%%%%%%%%%%%%%%%%

    %%% Definitions %%%%%%%%%%%%%%%%%%%%%%%%%%%%%%%%%%%%%%%%%%%%%%%%%%%%%%%%

%% \def\Low{\text{Low}}

    %%% Commands %%%%%%%%%%%%%%%%%%%%%%%%%%%%%%%%%%%%%%%%%%%%%%%%%%%%%%%%%%%

\newcommand{\D}[1]{D\p{#1}}


\newcommand{\rank}[2]{R\p{#1,#2}}
\newcommand{\rhat}[2]{\hat{R}\p{#1,#2}}
\newcommand{\rhats}[2]{\hat{R}_S\p{#1,#2}}
\newcommand{\rhatg}[2]{\hat{R}_G\p{#1,#2}}

\newcommand{\eva}[1]{\alpha_{#1}}
\newcommand{\evb}[1]{\beta_{#1}}



%%% Document data %%%%%%%%%%%%%%%%%%%%%%%%%%%%%%%%%%%%%%%%%%%%%%%%%%%%%%%%%%%%%%

\title{Implementation note for ``A randomized online quantile summary in
  $\OO{\frac{1}{\ep} \log \frac{1}{\ep}}$ words''\footnote{Felber, David, and
    Rafail Ostrovsky. "A Randomized Online Quantile Summary in O(1/epsilon* log
    (1/epsilon)) Words." LIPIcs-Leibniz International Proceedings in
    Informatics. Vol. 40. Schloss Dagstuhl-Leibniz-Zentrum fuer Informatik,
    2015.}}

\author{David Felber}
\date{2016 February 25--March 4}

    %%%%%%%%%%%%%%%%%%%%%%%%%%%%%%%%%%%%%%%%%%%%%%%%%%%%%%%%%%%%%%%%%%%%%%%%

\begin{document}

\maketitle

    %%%%%%%%%%%%%%%%%%%%%%%%%%%%%%%%%%%%%%%%%%%%%%%%%%%%%%%%%%%%%%%%%%%%%%%%

The bound in the referenced paper does not have particularly good constants.
This note improves those constants for common values of $\ep$ and $\delta$ (on
the order of $10^{-\p{1,2,3}}$).

\paragraph{Approach}
The main difference in the approach in this note and the one in the paper is
that we use mean deviation to bound the approximation error and then use
Markov's inequality to reduce the error probability to less than 1. Then, we run
$\OO{\logp{1/\delta}}$ copies in parallel, and on any query, we query all the
copies and return the median. This brings the probability of an incorrect query
response down to $\delta$.

\paragraph{Implementation}
We use a slightly different implementation. Here, row $i$ receives items up to
$2^i m$, so that row $0$ sees items $1 \ldots m$, row $1$ sees items $\p{m \pl
  1} \ldots 2m$, etc. We sample at rate $2^{-i}$, and fix each GK summary to
accept at most $2m$ items. In this note we assume that the probability that any
GK summary is ever offered more items is negligible.

\paragraph{Approximation}
The mean deviation $\D{u}$ for a variable $u$ is $E\abs{u - Eu}$. For $i \ge 1$,
we have that $\rank{y}{S_i}$ is distributed as $\text{Binomial}\p{n =
  \rank{y}{J_i}, p = 2^{-i}}$. Berend and Kontorovich state\footnote{Berend,
  Daniel, and Aryeh Kontorovich. "A sharp estimate of the binomial mean absolute
  deviation with applications." Statistics \& Probability Letters 83.4 (2013):
  1254-1259.} that for a $\text{Binomial}\p{n,p}$ variable the mean deviation is
at most the standard deviation, $\p{np\p{1-p}}$, which in our case we can bound
by $2^{-i/2} \rank{y}{J_i}^{1/2}$. Further, we know that $\rank{y}{J_i}^{1/2}
\le 2^i m$, so we can write $\D{\rank{y}{J_i}} \le 2^{i/2} m^{1/2}$.

Our internal GK summaries must use a finer approximation; we have them use an
error parameter $\ep/\alpha$, for some $\alpha > 1$. This means that
$\rank{y}{R_i} = \rank{y}{J_i} + z_i$ for some $\abs{z_i} \le 2^i \ep m /
\alpha$, from which we derive that $\D{\rank{y}{J_{i+1}} - \rank{y}{J_i}} \le
2^i \ep m / \alpha + 2^{i/2} m^{1/2}$. Summed over $i = 0 \ldots k$, where $k$
is the last full row, we get $\D{\rank{y}{J_{k+1}} - \rank{y}{J_0}} \le 2^{k+1}
\ep m / \alpha + 2^{\p{k+1}/2} m^{1/2}$.

\paragraph{Probability}
In order to use Markov's inequality to bound the failure probability to
$1/\beta$, we want
\begin{align*}
  \D{\rank{y}{J_{k+1}} - \rank{y}{J_0}} \le 2^k \ep m / \beta
\end{align*}
This means we want
\begin{align*}
  2^{k+1} \ep m / \alpha + 2^{\p{k+1}/2} m^{1/2} \le 2^k \ep m / \beta
\end{align*}
If we assume that $2^{\p{k+1}/2} \ep m^{1/2} / \alpha \ge 1$, then $\alpha \ge 2
\beta$ suffices. So if we take $\beta = 2$ (as we do in the code) then $\alpha =
4$ and $m = \p{\alpha / 2 \ep}^2 = 4 / \ep^2$ suffice.

        %%%%%%%%%%%%%%%%%%%%%%%%%%%%%%%%%%%%%%%%%%%%%%%%%%%%%%%%%%%%%%%%

\end{document}

% Copyright 2016 David Felber.
